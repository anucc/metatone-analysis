\documentclass{sigchi}


%% EXAMPLE BEGIN -- HOW TO OVERRIDE THE DEFAULT COPYRIGHT STRIP -- (July 22, 2013 - Paul Baumann)
% \toappear{Permission to make digital or hard copies of all or part of this work for personal or classroom use is      granted without fee provided that copies are not made or distributed for profit or commercial advantage and that copies bear this notice and the full citation on the first page. Copyrights for components of this work owned by others than ACM must be honored. Abstracting with credit is permitted. To copy otherwise, or republish, to post on servers or to redistribute to lists, requires prior specific permission and/or a fee. Request permissions from permissions@acm.org. \\
% {\emph{CHI'14}}, April 26--May 1, 2014, Toronto, Canada. \\
% Copyright \copyright~2014 ACM ISBN/14/04...\$15.00. \\
% DOI string from ACM form confirmation}
%% EXAMPLE END -- HOW TO OVERRIDE THE DEFAULT COPYRIGHT STRIP -- (July 22, 2013 - Paul Baumann)


% Arabic page numbers for submission.  Remove this line to eliminate
% page numbers for the camera ready copy 

\pagenumbering{arabic}

% Load basic packages
\usepackage{balance}  % to better equalize the last page
\usepackage{graphics} % for EPS, load graphicx instead 
%\usepackage[T1]{fontenc}
\usepackage{txfonts}
\usepackage{times}    % comment if you want LaTeX's default font
\usepackage[pdftex]{hyperref}
% \usepackage{url}      % llt: nicely formatted URLs
\usepackage{color}
\usepackage{textcomp}
\usepackage{booktabs}
\usepackage{ccicons}
\usepackage{todonotes}

% llt: Define a global style for URLs, rather that the default one
\makeatletter
\def\url@leostyle{%
  \@ifundefined{selectfont}{\def\UrlFont{\sf}}{\def\UrlFont{\small\bf\ttfamily}}}
\makeatother
\urlstyle{leo}

% To make various LaTeX processors do the right thing with page size.
\def\pprw{8.5in}
\def\pprh{11in}
\special{papersize=\pprw,\pprh}
\setlength{\paperwidth}{\pprw}
\setlength{\paperheight}{\pprh}
\setlength{\pdfpagewidth}{\pprw}
\setlength{\pdfpageheight}{\pprh}

% Make sure hyperref comes last of your loaded packages, to give it a
% fighting chance of not being over-written, since its job is to
% redefine many LaTeX commands.
\definecolor{linkColor}{RGB}{6,125,233}
\hypersetup{%
  pdftitle={SIGCHI Conference Proceedings Format},
  pdfauthor={LaTeX},
  pdfkeywords={SIGCHI, proceedings, archival format},
  bookmarksnumbered,
  pdfstartview={FitH},
  colorlinks,
  citecolor=black,
  filecolor=black,
  linkcolor=black,
  urlcolor=linkColor,
  breaklinks=true,
}

% create a shortcut to typeset table headings
% \newcommand\tabhead[1]{\small\textbf{#1}}

% End of preamble. Here it comes the document.
\begin{document}

\title{Quantitative Evaluation of Touch-Screen Performance with
  Transition Matrix Measures}

\numberofauthors{3}
\author{%
  \alignauthor{1st Author Name\\
    \affaddr{Affiliation}\\
    \affaddr{City, Country}\\
    \email{e-mail address}}\\
  \alignauthor{2nd Author Name\\
    \affaddr{Affiliation}\\
    \affaddr{City, Country}\\
    \email{e-mail address}}\\
  \alignauthor{3rd Author Name\\
    \affaddr{Affiliation}\\
    \affaddr{City, Country}\\
    \email{e-mail address}}\\
}

\maketitle

\begin{abstract}
  HCI systems for supporting collaborative creativity are often
  evaluated with user ratings and interviews. However, such data can
  be impractical to collect in-the-wild, such as during musical
  performances. In this paper we consider measures of gestural
  transitions during collaborative interactions with musical touch
  screen interfaces. The performers' interactions are characterised as
  sequences of discrete gestures and summarised in a transition matrix
  for each performance. Using a corpus of more than 140 performances,
  rehearsals, and demonstrations, we find that two matrix measures,
  flux and entropy, vary significantly with different kinds of
  in-the-wild interactions. Further, using performers' ratings
  collected during a lab-study we find that these measures have a
  significant relationship with the perceived effect of the interface
  on performance. Thus transition analysis, flux and entropy, can be
  used as a proxy for performer ratings during in-the-wild interaction.
\end{abstract}

\keywords{  Creativity Support Tools;
  Agent; 
  Design;
  Methodology}

\category{H.5.5.}{Information Interfaces and Presentation
  (e.g. HCI)}{Sound and Music Computing}{Systems} 

% \category{See
%   \url{http://acm.org/about/class/1998/} for the full list of ACM
%   classifiers. This section is required.}{}{}

\section{Introduction}



Finally, we report a quantitative analysis of the transition matrices
from our experimental protocols, and describe the quantities
\emph{flux} and \emph{entropy} which can be used to characterise the
gestural activity of the musicians during a performance. We found a
significant effect of the app on these measures, allowing us to
observe quantitatively the performers' dissatisfaction with a
particular interface. We also investigate the significant relationship
between both of these measures and the survey responses, which
suggests that performers rate interaction with our agent and apps more
highly when engaged in adventurous, exploratory performance.

\section{Related Work}

\section{Transition Matrices and Matrix Measures}

It has been recognised that collaborative interactions such as
improvised musical performances can be modelled as a sequence of
states. In the following section we present a method for analysing the
transitions between adjacent states in such a sequence and derive two
high-level quantities that we use to compare musical performances
under different circumstances. Suppose that an collaborative
interaction can be characterised by a set of interaction states $S$
with $m$ members.
Each participant's activity can then be represented as a sequence:
\begin{equation}
 X_n \hskip 2em n = 1, \ldots, N
\end{equation}
Where each $X_i$ is a member of $G$. The transitions between different
interaction states can be summarised in a transition matrix as if the
sequence $X_n$ were a first-order Markov chain. Let $N_{ij}$ be the
number of times that state $j$ follows state $i$. Each participant's
transition matrix, $P$, will then be an $m \times m$ matrix with
entries calculated thus
\begin{equation}
  p_{ij} = \frac{N_{ij}}{\sum_j N_{ij}}
\end{equation}
Each entry of $P$ will be such that the transition probability from
state $i$ to state $j$ is given by the entry in the $i^{th}$ row and
the $j^{th}$ column. The transition activity of the whole ensemble can
be calculated by averaging the transition matrix for each performer.

To compare the transition matrices of different interactions we use
two high-level quantities, ``flux'' and ``entropy''. The flux of a
transition matrix $P$ is given by
\begin{equation}
  \mathrm{flux}(P) = \frac{\|P\|_1-\|\mathrm{diag}(P)\|}{\|P\|_1}
\end{equation}
Where $\|P\|_1 = \sum_{i,j}|p_{ij}|$ is the element-wise 1-norm of $P$
and $\mathrm{diag}(P)$ is the vector of the main diagonal entries of
$P$.

Flux measures the ratio of transitions that change
state to self-transitions from a certain state back to itself.
Intuitively, it is a measure of how frequently a group changes state. Flux
returns a value in the interval $[0,1]$ and will return 0 when
participants never change state, and 1 when participants never stay on
the same state for two measurements in a row.

A related (but different) high-level derived quantity for the
transition matrix is its entropy, defined in the
information-theoretic~\cite{Shannon:1948rt} sense:
\begin{equation}
  H(P) = -\sum_{i,j}p_{ij}\log_2(p_{ij})
\end{equation}
This entropy measure is small when the matrix is sparse, and largest
when each matrix element is equal. It offers a different perspective
on collaborative interactions than the flux measure by capturing the
breadth of the gestural space explored by the participants throughout
the course of an interaction. Consider the degenerate case of a
participant who alternates between two states for a whole performance:
the flux in this case will be maximal ($\mathrm{flux}(P) = 1$) since
there are no self-transitions, (only $A \rightarrow B$ or
$ B \rightarrow A$) even though the participant has only used a small
subset of the state space. The entropy of this interaction,
on the other hand, will be low. Entropy, therefore, is a measure of
how broad the participant's exploration of the state space is in a given
interaction.

\section{A Corpus of Musical Touch-Screen Interactions}

In the following sections, we will use transition matrices, flux, and
entropy to analyse a corpus of data collected as part of our research
into collaborative interactions with musical touch-screen tablet
interfaces. In a period of more than two years (April 2013 - June 2015), we have
collected data from $135$ sessions where various participants used
musical touch-screen interfaces on Apple iPad devices. These sessions
had a number of contexts including rehearsals, performances, lab study
sessions, classes, and casual demonstrations. Participants performed
free improvisations as well as written compositions, performed on
iPads alone and in setups of iPads and acoustic percussion
instruments. The median length of these sessions was 7:32 and the
median number of participants was four.

\section{Modelling Performer Ratings with Flux and Entropy}

\begin{figure}
  \centering
  \includegraphics[width=\linewidth]{figures/stat-response-fit.pdf}
  \caption{Transition matrix statistics (flux above and entropy below) vs Likert responses for
    questions 3, 6 and 7, all of which exhibited significant
    relationships under a proportional odds logistic regression model.
    A least-squares linear model is shown with a dotted line.
    \label{fig:stat-response-fit}}
\end{figure}

One final question to explore is how these flux and entropy measures
in the interaction data relate to the performers' perceptions of their
performance?

To answer this question, we construct models with the flux and entropy
measures as independent variables and the Likert responses (1 to 5) as
the dependent variable. These measures exhibited clear relationships
for three of the survey questions (Questions 3, 6 and 7), which are
plotted in Figure~\ref{fig:stat-response-fit} where the least-squares
line of best fit has a positive gradient for both flux and entropy.

Determining the strengths of these positive relationships is
difficult, both due to the problem of treating ordinal Likert
responses as numeric~\cite{Gardner:2007fj} and also
due to the unknown distribution of the entropy and flux statistics. As
a result, we use a conservative proportional odds logistic regression
(POLR)~(using the \texttt{polr} function in \texttt{R}, from the MASS
package \cite{Venables:2002qv}) to treat the Likert responses as an
ordered factor response.

The POLR results indicate a significant ($p<0.05$) relationship
between the user-reported Likert response and the flux statistic for
Question 3 ($t = 2.77, df = 4, p = 0.025$), Question 6
($t = 2.41, df = 4, p = 0.037$), and Question 7
($t = 2.49, df = 4, p = 0.034$). The same questions displayed a
significant relationship for the entropy statistic: Question 3
($t = 2.93, df = 4, p = 0.022$), Question 6
($t = 2.69, df = 4, p = 0.028$), and Question 7
($t = 2.89, df = 4, p = 0.022$). These relationships are shown in
Figure~\ref{fig:stat-response-fit}.

It is notable that these questions (3, 6, and 7) all pertain to the
\emph{perceived effect of the agent} on the performers. As the agent
uses the flux measure to trigger new-idea messages, it appears that
performers appreciated its intervention more in particularly creative
performances where flux and entropy were high.






\bibliographystyle{SIGCHI-Reference-Format}
\bibliography{references}

\end{document}

%%% Local Variables:
%%% mode: latex
%%% TeX-master: t
%%% End:
